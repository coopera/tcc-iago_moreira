% Pre-ambulo
\documentclass[a4paper, 12pt]{abnt}

\usepackage[brazil]{babel}
\usepackage[T1]{fontenc}
\usepackage[utf8]{inputenc}
\usepackage{dsfont}
\usepackage{amssymb,amsmath}
\usepackage{multirow}
\usepackage[alf]{abntcite}
\usepackage[pdftex]{color, graphicx}
\usepackage{colortbl}
\usepackage{url}
\usepackage{abnt-alf}
\usepackage{abntcite}
\usepackage{algorithm}
\usepackage{algorithmic}

%\usepackage{alg}


% Redefinicao de instrucoes
\floatname{algorithm}{Algoritmo}
\renewcommand{\algorithmicrequire}{\textbf{Entrada:}}
\renewcommand{\algorithmicensure}{\textbf{Saída:}}
\renewcommand{\algorithmicend}{\textbf{fim}}
\renewcommand{\algorithmicif}{\textbf{se}}
\renewcommand{\algorithmicthen}{\textbf{então}}
\renewcommand{\algorithmicelse}{\textbf{senão}}
\renewcommand{\algorithmicfor}{\textbf{para}}
\renewcommand{\algorithmicforall}{\textbf{para todo}}
\renewcommand{\algorithmicdo}{\textbf{faça}}
\renewcommand{\algorithmicwhile}{\textbf{enquanto}}
\renewcommand{\algorithmicloop}{\textbf{loop}}
\renewcommand{\algorithmicrepeat}{\textbf{repetir}}
\renewcommand{\algorithmicuntil}{\textbf{até que}}
\renewcommand{\algorithmiccomment}[1]{\% #1}


\newcommand{\ignore}[1]{}


% Hifenização de palavras feita de forma incorreta pelo LaTeX
\hyphenation{PYTHON ou-tros}


% Inicio do documento
\begin{document}

	\frenchspacing
	
	% Capa (arquivo Includes/Capa.tex)
	% Capa
% Proteção externa do trabalho e sobre a qual se imprimem as informações indispensáveis 
% à sua identificação.

% Especificação da capa
\begin{titlepage}
	\begin{center}
		
		% Cabeçalho (não deve ser modificado)
		% Contém o brasão da Universidade, o logotipo do Departamento, além dos dados
		% relacionados à vinculação do aluno (Universidade, Centro, Departamento e Curso)
		\begin{minipage}{2cm}
			\begin{center}
				\includegraphics[width=1.7cm, height=2.0cm]{Imagens/Brasao-UFRN.jpg}
			\end{center}
		\end{minipage}
		\begin{minipage}{11cm}
			\begin{center}
				\begin{espacosimples}
					{\small \textsc{Universidade Federal do Rio Grande do Norte}			\\
							  \textsc{Centro de Ciências Exatas e da Terra}						\\
							  \textsc{Departamento de Informática e Matemática Aplicada}	\\
							  \textsc{Bacharelado em Engenharia de Software}}
				\end{espacosimples}
			\end{center}
		\end{minipage}
		\begin{minipage}{2cm}
			\begin{center}
				\includegraphics[width=1.8cm, height=1.5cm]{Imagens/Logotipo-DIMAp.jpg}
			\end{center}
		\end{minipage}
			
		\vspace{5cm}
						
		% Título do trabalho
		{\setlength{\baselineskip}%
		{1.3\baselineskip}
		{\LARGE \textbf{Estudo e Implementação de uma Ferramenta de Documentação em Suporte à Reuso de Conhecimento em Equipes de Desenvolvimento de Software}}\par}
			
		\vspace{4cm}
			
		% Nome do aluno (autor)
		{\large \textbf{Igor Marques da Silva}}
						
		\vspace{7cm}
		
		% Local da instituição onde o trabalho deve ser apresentado e ano de entrega do mesmo
		Natal-RN\\Dezembro 2014
	\end{center}
\end{titlepage}

	% Folha de rosto (arquivo Includes/FolhaRosto.tex)
	% Folha de rosto
% Contém os elementos essenciais à identificação do trabalho.

% Título, nome do aluno e respectivo orientador e filiação
\titulo{\Large{Estudo e Implementação de uma Ferramenta de Documentação em Suporte à Reuso de Conhecimento em Equipes de Desenvolvimento de Software}}
\autor{Igor Marques da Silva}
\orientador[Orientador]{\par Prof. Dr. Fernando Marques Figueira Filho}
\instituicao
{
	Universidade Federal do Rio Grande do Norte -- UFRN \par 
	Departamento de Informática e Matemática Aplicada -- DIMAp
}
	
% Natureza do trabalho (não deve ser modificada)
\comentario
{
	Proposta de Monografia de Graduação apresentada ao Departamento de Informática e Matemática Aplicada do 
	Centro de Ciências Exatas e da Terra da Universidade Federal do Rio Grande do Norte como
	requisito parcial para a obtenção do grau de bacharel em Engenharia de Software.
}
		
% Local e data
\local{Natal-RN}
\data{Dezembro de 2014}
	
\folhaderosto	
	
	% Folha de aprovacao (arquivo Includes/FolhaAprovacao.tex)
	% Folha de aprovação
\begin{folhadeaprovacao}
	\setlength{\ABNTsignthickness}{0.4pt}
	\setlength{\ABNTsignwidth}{10cm}

	% Informações gerais acerca do trabalho
	% (nome do autor, título, instituição à qual é submetido e natureza)
	\noindent
	Monografia de Graduação sob o título \textit{Twydi: Uma Ferramenta de Documentação para Equipes de Desenvolvimento de Software} apresentada por
	Igor Marques da Silva e aceita pelo Departamento de Informática e Matemática Aplicada do
	Centro de Ciências Exatas e da Terra da Universidade Federal do Rio Grande do Norte,
	sendo aprovada por todos os membros da banca examinadora abaixo especificada:

	% Membros da banca examinadora e respectivas filiações
	\assinatura
	{
		Prof. Dr. Fernando Marques Figueira Filho\\
		{\small Orientador} 															\\
		{\footnotesize
			Departamento de Informática e Matemática Aplicada 																	\\
		  	UFRN
		}
	}

	\assinatura
	{
		Prof. Dr. Gibeon Soares De Aquino Junior						 \\
		{\footnotesize
			Departamento de Informática e Matemática Aplicada 																	\\
		  	UFRN
		}
	}

	\assinatura
	{
		Profa. Dra. Marcia Jacyntha Nunes Rodrigues Lucena						 \\
		{\footnotesize
			Departamento de Informática e Matemática Aplicada 																	\\
		  	UFRN
		}
	}

	\vfill

	\begin{center}
		Natal-RN, sete de dezembro de dois mil e quize
	\end{center}
\end{folhadeaprovacao}
	
	
	% Dedicatoria (arquivo Includes/Dedicatoria.tex)
	% Dedicatória

\chapter*{}
\vspace{15cm}
\begin{flushright}
	À todos que fizeram isso possível:

	Meus pais, \textbf{Roberto} e \textbf{Marcia}, por todo amor, educação e oportunidades que me deram ao longo da minha vida.

	Meu irmão, \textbf{Ivan}, por alegrar os meus dias com o seu jeito de ser.

	Minha querida \textbf{Mayane}, por todo amor e carinho além do que um ser humano pode receber.

\end{flushright}

	
	% Agradecimentos (arquivo Includes/Agradecimentos.tex)
	% Agradecimentos

\chapter*{Agradecimentos}

Em quatro anos de graduação, muito amadureci e muito aprendi. Sem dúvida tudo isso não seria possível sem uma boa parcela de companheirismo, apoio e dedicação. Tudo o que alcancei foi graças a uma boa parcela de pessoas.

Obrigado à minha família, por tanto apoio. Não seria nada sem vocês. Literalmente. Obrigado por dia após dia me aturarem, me educarem e me proverem com tudo que há de melhor nesse mundo.

Obrigado à minha companheira de todas as horas. Obrigado por tanta paz, tanto carinho e tanto amor todos os dias.

Agradeço também a todos os meus amigos de ensino médio da querida turma 401 do IFRN, porque vocês merecem mesmo não tendo nada a ver com esse trabalho.

Devo agradecer também a todos que me ajudaram a adquirir tantos ensinamentos ao longo desses quatro anos. Sei o que sei hoje da área de computação graças a uma grande parcela de amigos, colegas e professores.

Obrigado especialmente à 4Soft e a todos que fizeram ou fazem parte dela. Não seria o profissional que sou hoje sem essa tão caristmática (e problemática) empresa júnior. Foram três anos e meio de muita luta, aprendizados e ódio acumulado (por fatores que não vem ao caso) e que sem dúvida abriram meus olhos com relação ao mundo.

Um agradecimento especial também aos meus companheiros de curso que mais me ensinaram nesse tempo todo (e que certamente continuarão ensinando). Vou citá-los em ordem alfabética para que ninguém fique com ciúmes, ok? Obrigado Bernardo, Iago, Lucas, Luiz Rogério e Waldyr por tantas dicas trocadas, \textit{pulls requests} revisados, dúvidas esclarecidas e trabalhos feitos.

Agradeço também a todos os professores do Departamento de Informática e Matemática Aplicada da UFRN, em especial ao professor Fernando, pelas infindáveis orientações e revisões deste trabalho.Obrigado também ao professor Nélio Cacho por ter passado o trabalho mais desafiador na disciplina mais insana que paguei durante a graduação e que serviu de um amadurecimento enorme.

Obrigado também a qualquer um que se sinta válido de estar mencionado aqui mas acabou não sendo por algum motivo qualquer. Desculpa.

Obrigado a todos que torceram por mim.

Obrigado.

   
   % Epigrafe (arquivo Includes/Epigrafe.tex)
	% Epígrafe (citação seguida de indicação de autoria)

\chapter*{}
\vspace{15cm}
\begin{flushright}
	\textit
	{
		Life moves pretty fast. If you don't stop and look around once in a while, you could miss it.
	}\medskip\\
	Ferris Bueller
\end{flushright}

	
	% Resumo em língua vernacula (arquivo Includes/Resumo.tex)
	% Resumo em língua vernácula
\begin{center}
	{\Large{\textbf{Twydi: Uma Ferramenta de Documentação para Equipes de Desenvolvimento de Software}}}
\end{center}

\vspace{1cm}

\begin{flushright}
	Autor: Igor Marques da Silva\\
	Orientador: Prof. Dr. Fernando Marques Figueira Filho
\end{flushright}

\vspace{1cm}

\begin{center}
	\Large{\textsc{\textbf{Resumo}}}
\end{center}

\noindent Desenvolvedores de software constantemente trocam informações entre si para o aprimoramento contínuo de suas habilidades e resolução de problemas. Muitas dessas informações são perdidas devido ao uso de mídias que não oferecem bom suporte ao armazenamento, recuperação e visualização de informações.
Este trabalho apresenta uma ferramenta que permite aos desenvolvedores de agregar descrições textuais, código, referências externas em artefatos, além de possibilitar a recuperação destes, bem como os estudos realizados ao longo do seu processo de desenvolvimento e avaliação. Conclui que, apesar de melhorias no quesito de pré-visualização dos artefatos serem necessárias, a ferramenta atende os requisitos de suporte a troca de conhecimento de desenvolvedores através do armazenamento do registro de soluções e compartilhamento de informações e  referências para a equipe. Provê uma base para futuras ferramentas que se proponham a resolver os mesmos problemas.

\noindent\textit{Palavras-chave}: Documentação de Software. Reuso de Software. Desenvolvimento de Software.

	
	% Abstract, resumo em língua estrangeira (arquivo Include/Abstract.tex)
	% Resumo em língua estrangeira (em inglês Abstract, em espanhol Resumen, em francês Résumé)
\begin{center}
	{\Large{\textbf{Twydi: A tool for Documentation for Software Development Teams}}}
\end{center}

\vspace{1cm}

\begin{flushright}
	Author: Igo Marques da Silva\\
	Advisor: Prof. Dr. Fernando Marques Figueira Filho
\end{flushright}

\vspace{1cm}

\begin{center}
	\Large{\textsc{\textbf{Abstract}}}
\end{center}

\noindent Software Developers exchange information all the time to improve skills and solve problems. Many of this informations are lost due to the use of medias that don't offer support to information storage, recovery and visualization.

This work presents a tool that allows developers to combine text, code and external references to artifacts and later its retrieving. It also presents studies during its development and evaluation. Even though improvements related to arctifacts preview be needed, it supports knowledge sharing between developers through the storage of solutions, information and references. It provides a startup point for future tools with the same purpose.

\noindent\textit{Keywords}: Software Documentation. Software Reuse. Software Development.

	
	% Lista de figuras
% 	\listoffigures

	% Lista de tabelas
% 	\listoftables
	
	% Lista de abreviaturas e siglas
% 	\listadeabreviaturas
	
	% Lista de símbolos
% 	\listadesimboenlos
	
	% Lista de algoritmos (se houver)
	% Devem ser incluídos os pacotes algorithm e algorithmic
	% \listofalgorithms
	
	% Sumário
	\sumario

	% Parte central do trabalho, englobando os capítulos que constituem o mesmo
	% Os referidos capítulos devem ser organizados dentro do diretório "Capítulos"

	\chapter{Introdução}

%ORIGINAL
% No contexto de trabalhadores do conhecimento, como na área de desenvolvimento de software, é diária a troca de conhecimento com o objetivo de se obter um melhor desempenho para a organização inteira~\cite{Druker1993, Wiig2003}. A gerência de conhecimento lida com o reuso de conhecimento em  suas diferentes formas, como: design de código, requisitos, modelos, dados, padrões e lições aprendidas~\cite{Levy2009}. Um dos casos de reuso é o conhecimento de como se deu a implementação de uma determinada funcionalidade em um determinado projeto de software. Exemplo: diferentes projetos podem implementar cadastro de usuário via Google ou Facebook, cada um com certas especificidades.


%FERNANDO 25/08
Desenvolvedores de software são trabalhadores do conhecimento e frequentemente precisam reutilizar o conhecimento aprendido por outros desenvolvedores, de modo a proporcionar um melhor desempenho para toda organização~\cite{Druker1993}~\cite{Wiig2003}.\ignore{FONTE 1 - COMEÇO} É exigido diaramente dos trabalhadores do conhecimento a melhoria contínua de seu trabalho em um processo em culmina na melhora significativa da sua empresa como um todo. [PEGAR FONTE 6][PEGAR FONTE 21]\ignore{FONTE 1 - FIM}A gerência de conhecimento lida com o reuso de conhecimento em  suas diferentes formas, como: design de código, requisitos, modelos, dados, padrões e lições aprendidas~\cite{Levy2009}\ignore{FONTE 2 - COMEÇO} A gerência de conhecimento está encarregada da elicitação, armazenamento e gerenciamento e reuso do conhecimento em suas diferentes formas, em particular, artefatos da engenharia de software como código, arquitetura, requisitos, modelos, dados, padrões e lições aprendidas. Desenvolvimento de software pode ser melhorando se reconhecendo o conteúdo e estrutura do conhecimento relacionado bem como de conhecimento necessário, além de atividades de planejamento. PEGAR FONTE]
\ignore{FONTE 2 - FIM}. Dentre esse reuso, se encontra, por exemplo, como se deu a implementação de uma determinada funcionalidade em um determinado projeto de software.

% FONTE 1 - Knowledge workers are required to improve their work on a daily basis in a process that cumulates into a significant improvement in performance for the entire enterprise [6][21].
% IM - TRADUÇÃO:
% É exigido diaramente dos trabalhadores do conhecimento a melhoria contínua de seu trabalho em um processo em culmina na melhora significativa da sua empresa como um todo

% FONTE 2 - KM is comprised of the elicitation, packaging and management, and reuse of knowledge in all of its different forms, and in particular, software engineering artifacts as code, design, requirements, models, data, standards, and lessons learned.
%Software development can be improved by recognizing the related knowledge content and structure as well as the required knowledge, and performing planning activities. - 05071412.pdf
% IM - TRADUÇÃO:
% A gerência de conhecimento está encarregada da elicitação, armazenamento e gerenciamento e reuso do conhecimento em suas diferentes formas, em particular, artefatos da engenharia de software como código, arquitetura, requisitos, modelos, dados, padrões e lições aprendidas.
% Desenvolvimento de software pode ser melhorando se reconhecendo o conteúdo e estrutura do conhecimento relacionado bem como de conhecimento necessário, além de atividades de planejamento.

%-------

% ORIGINAL
% É comum também que essa implementação em diferentes contextos seja feita através de abordagens \textit{ad-hoc} (reimplementação completa de trechos de código com poucas modificações e sem modularizacão visando reuso)~\cite{SangMok2011}. Nesse caso, desenvolvedores mais experientes em um determinado projeto ou que já implementou tal funcionalidade tendem a atuar como mentores~\cite{CubraniC2004} e tal ato, como atividade de gerência de conhecimento, acarreta no despendimento de recursos, principalmente dos mentores~\cite{Wiig2003}.
%
% Atualmente, as maneiras mais comuns de troca de informações entre desenvolvedores (incluindo, obviamente, a mentoria) são via oral, escrita ou repasse de referências (documentação, links externos, etc)~\cite{Storey2014, Olson2000, CubraniC2004}. Em alguns casos, pela própria natureza do meio, comunicação não se mantém registrada física ou virtualmente. Um dos exemplos de comunicação sem registro é a comunicação exclusivamente oral em que, ao fim do diálogo, as informações trocadas ficam apenas na memória dos envolvidos, sem maneiras de terceiros consultarem posteriormente aquelas informações~\cite{Olson2000}.

A troca de informações é um mecanismo fundamental para que o reuso de conhecimento ocorra de maneira eficiente em equipes de desenvolvimento. Atualmente, desenvolvedores utilizam diversos meios de comunicação para trocar informações, através de interações face a face, comunicação escrita e também via o repasse de referências de documentação, links externos, dentre outros.~\cite{Storey2014}~\cite{Olson2000}~\cite{CubraniC2004}\ignore{FONTE 3}
% FONTE 3 - p100-storey.pdf

% FF: Além disso, insira aqui um parágrafo de exemplo. Pense numa situação (pode ser hipotética) em que a falta de um registro sobre como fazer alguma coisa dificultou o processo de desenvolvimento de alguma maneira.
% IM: está bom?
% ||
% \/

Desenvolvedores em fase de aprendizado de uma determinada tecnologia comumente passam por situações e, em decorrência,  problemas semelhantes. Quando se deparam com algum tipo de adversidade, procuram por alguma fonte de informação capaz de auxiliá-los por tal problema. Existe então o desperdício de tempo de se recuperar tal informação (muitas vezes, interferindo nas atividades de um colega de trabalho) para se resolver um problema que já é de conhecimento da equipe dado alguma experiência anterior.

% IGOR: REVISAR A PARTIR DAQUI

É comum também desenvolvedores implementarem funcionalidades semelhantes em diferentes contextos usando abordagens ad-hoc~\cite{SangMok2011}.
% FF: A sentença acima não está clara.
Nesse caso, desenvolvedores mais experientes em um determinado projeto tendem a atuar como mentores~\cite{CubraniC2004} e tal ato, como atividade de gerência de conhecimento, tende a consumir recursos~\cite{Wiig2003}.\ignore{FONTE 4} Muitas das interações entre desenvolvedores são informais e não há registro que possa ser consultado para propocionar o reuso do conhecimento trocado~\cite{Olson2000}. Um agravante tipicamente é a frequente rotatividade de membros em equipes. Nesses casos, a saída de um membro da equipe que detém determinado conhecimento pode prejudicar toda organização. De fato, a falta de uma abordagem mais sistemática para proporcionar o reuso de conhecimento pode estar associada ao fracasso de projetos em organizações~\cite{Hall2008}.
% FONTE 4 - One of the cornerstones of KM is improving productivity by effective sharing and transfer of knowledge, which tends to be time-consuming and often impossible [21] - 05071412.pdf.

%-------


Além disso, existe o fator da rotatividade de membros em equipes, sempre associado a custos de transferência de conhecimento e treinamento~\cite{Hall2008}, possuindo relação com sucesso ou fracasso de projetos de software~\cite{Hall2008}.

Assim, a elaboração de uma ferramenta capaz de agregar referências de código e tarefas a soluções pode trazer enormes benefícios a equipes de desenvolvimento~\cite{CubraniC2004}. A ferramenta atua como um catálogo, agregando referências de código, informações externas e comentários fornecidos por desenvolvedores da equipe com o intuito de auxiliar outros desenvolvedores a buscar em fontes da própria equipe como se deram implementações de funcionalidades semelhantes às que já foram feitas em outros projetos.

Segue um exemplo de uso da ferramenta: Um determinado desenvolvedor X, experiente dentro de sua equipe, percebe que muitos membros novatos procuram sua ajuda para implementar uma funcionalidade de exportação de uma página web para um documento em formato PDF. X então pode fazer uso da ferramenta, para registrar como se realiza a implementação de tal funcionalidade solicitada. Ele informa um título, descrição curta (para facilitar futuras buscas) e uma descrição de como se dá tal implementação. Como ilustração em formato de código, o desenvolvedor pode utilizar de links de trechos de código disponívels artefatos de código já implementados pela sua equipe. Esses links são renderizados no editor de texto, sem a necessidade de X copiar e colar o código em si dentro do editor. X também pode complementar sua descrição adicionando links para outras páginas web (respostas de sites como Stack Overflow, outros tutoriais, etc...) em forma de anexo da documentação. Por fim, X informa tags relevantes ao artefato que está produzindo, facilitando a sua recuperação por parte de outros membros de sua equipe.

Outros desenolvedores, a partir de então, podem recorrer diretamente a ferramenta quando precisarem implementar a funcionalidade de exportação de página em formato PDF. Dessa forma, X poderá se dedicar mais a outras atividades de seu dia-a-dia de trabalho.

Este estudo descreve a elaboração de tal ferramenta com a participação de uma equipe real de desenvolvimento de software. A equipe em questão é a 4Soft\footnote{\url{http://www.4softjr.com.br}}, empresa júnior\footnote{\url{http://en.wikipedia.org/wiki/Junior_enterprise}} dos cursos de Engenharia de Software e Tecnologia da Informação da Universidade Federal do Rio Grande do Norte (UFRN\footnote{\url{http://www.ufrn.br}}). A empresa atua na área de desenvolvimento de software Web para clientes de diversos ramos e é formada exclusivamente por alunos dos cursos de Bacharelado em Engenharia de Software\footnote{\url{http://www.dimap.ufrn.br/pt/graduacao/engenharia-de-software/apresentacao}} e Bacharelado em Tecnologia da Informação\footnote{\url{http://www.imd.ufrn.br/curso_bacharelado.php}} da UFRN.

O estudo também realiza uma análise crítica dos impactos do uso da mesma no contexto de empresa. Dentre os impactos previstos estão a redução da necessidade de mentoria ou consulta para os casos de  reuso de funcionalidades em diferentes contextos. Tais impactos visam contribuir para a redução de gastos de recursos relacionados a tempo de implementação de funcionalidades, esforço de recuperação de informações, orientação de colaboradores e trabalho dos desenvolvedores que atuam como mentores dentro da equipe.

% falta adaptar para a estrutura do TCC

% O Capítulo 2 descreve em detalhes o objetivo pretendido pelo trabalho e a quais perguntas pretende-se responder. Em seguida, o Capítulo 3 aborda a metodologia utilizada para a execução do estudo e que atividades serão desempenhadas pelos envolvidos. O Capítulo 4, por fim, descreve o cronograma das atividades descritas no capítulo anterior.

	
	\chapter{Objetivos}

% gerais e específicos
Este trabalho tem por finalidade analisar que influência uma ferramenta de documentação de funcionalidades de software traz para equipes de desenvolvimento. Além disso, objetiva verificar como tal ferramenta pode contribuir para a redução significativa de tempo despendido para a explicação repetitiva de tarefas já executadas anteriormente e como a troca de conhecimento entre membros pode fluir de maneira melhor e com baixo custo.

% A ferramenta está prevista de possuir os seguintes requisitos:

% \begin{enumerate}
% \item Gerenciamento de catálogo de funcionalidades implementadas em um projeto com integração com repositório de código no GitHub
% \item Anexo de fontes externas que auxiliem a implementação de determinada funcionalidade
% \item Anexo de descrição de etapas da implementação de determinada funcionalidade
% \item Recuperação de funcionalidades no catálogo
% \item Anexo de variações da implementação de determinada funcionalidade
% \end{enumerate}

% Assim, tal ferramenta prevê inicialmente os benefícios de fácil recuperação de referências sem a necessidade de intermédio de colegas de trabalho e, por consequência, garantindo que menos tempo será demandado para a orientação de desenvolvedores.

Este trabalho visa então responder às seguintes perguntas de pesquisa:

\begin{enumerate}
\item Quais são os requisitos para o suporte ferramental em apoio à transferência e reuso do conhecimento organizacional em empresas de desenvolvimento de software?
\item Que outras ferramentas existentes oferecem tal suporte?
\item Como estimular a adoção e o uso desse suporte ferramental em equipes de desenvolvimento de software? 
\item Que benefícios e limitações tal suporte provê a equipes de desenvolvimento?

\end{enumerate}


   
	
	\chapter{Métodos}

% O que será feito, como será feito, … para abordar o problema

Este trabalho prevê o desenvolvimento da ferramenta de documentação de implementação funcionalidades. Nela, será possível vincular recursos do repositório do projeto no GitHub\footnote{\url{www.github.com}} (\textit{issues}, \textit{commits}, etc) a requisitos e suas implementações, bem como outros referenciais (\textit{links} para perguntas no Stack Overflow\footnote{\url{www.stackoverflow.com}}, desenhos, por exemplo) de forma a gerar um guia ou tutorial de como realizar tal implementação novamente no futuro.

A empresa 4Soft terá participação significativa em todo o estudo, desde a concepção até no uso da ferramenta.

% metodo: entrevista, questionário de satisfacao. Focar em pensar em como responder as perguntas

As etapas do estudo são:


\begin{description}
   \item Estudo de aplicações existentes

		Será feita uma busca por aplicações que realizem atividades semelhantes às propostas. Suas limitações serão analisadas pela equipe de pesquisa e um panorama inicial será traçado de modo que a ferramenta proposta possa suprir as necessidades iniciais e as limitações encontradas.
    
    \item Inquérito contextual

		Entrevistas e seções de \textit{brainstorm} serão feitas com os participantes da empresa júnior mencionada. Será analisado como se dá seu processo de trabalho e como pode se dar o fluxo de atividades na ferramenta. As informações necessárias serão coletadas através de entrevistas e aplicação de questionários de satisfação.

	\item Sessões de interpretação da equipe

		Reuniões com a equipe de pesquisadores que trabalharão no projeto serão feitas para definir com mais detalhes o escopo da ferramenta, bem como será sua arquitetura e implementação.

	\item Prototipação e implementação da ferramenta

		Nesta etapa, inicialmente, protótipos de baixa fidelidade serão elaborados pelo pesquisador. Posteriormente, serão expostos a todos os participantes do projeto (empresa júnior e pesquisadores) e seu \textit{feedback} será colhido e analisado. A partir daí, a aplicação passará para a etapa de implementação seguindo processo iterativo e incremental de desenvolvimento de software.

	\item Implantação e observação do uso da ferramenta

		A ferramenta então estará disponível para uso de todos os membros da empresa 4Soft. A adesão dos desenvolvedores a ferramenta será analisada nesta etapa, bem como seu uso monitorado (quantidade de artefatos de documentação criados, por exemplo).
        
        
	\item Avaliação da ferramenta

		Ao fim do período anterior, uma nova bateria de entrevistas e aplicação de questionários de satisfação serão realizadas para avaliar qualitativamente como se deu a utilização da ferramenta, bem como se deram os efeitos de seu uso e a satisfação de seus usuários. 
		A versão final do trabalho com os resultados será então redigida.


\end{description}
	
	\chapter{Plano de Trabalho}

% Tarefas ou Atividades (bem macro)
% Cronograma

O cronograma de atividades se dará conforme o seguinte planejamento:

\begin{enumerate}

\item Estudo de aplicações existentes: Dezembro/2014

\item Inquérito contextual: Janeiro/2015

\item Sessões de interpretação da equipe: Janeiro/2015

\item Prototipação e implementação da ferramenta: Janeiro/2015 a Maio/2015

\item Implantação e observação do uso da ferramenta: Junho/2015 a Setembro/2015

\item Avaliação da ferramenta: Outubro/2015 a Novembro/2015

\end{enumerate}

		

	% Bibliografia (arquivo Capitulos/Referencias.bib)
	\bibliography{Capitulos/Referencias}
	\bibliographystyle{abnt-alf}

	% Página em branco
	\newpage

\end{document}