\chapter{Revisão de Literatura}

\section{Uso das mídias pelo engenheiro de software}

A atividade de desenvolvimento de software requer que seus profissionais se mantenham constatemente atualizados, devido ao fluxo constante de inovações que surgem diariamente. [CARECE DE FONTES]
% Google Citations:
% 1 - Revolution
% 2 - Speed of Light Twitter
% 3 - Social Programmer.pdf

Uma das formas que desenvolvedores utilizam para se manterem atualizados são as redes sociais [CARECE DE FONTES] além de seus próprios colegas de trabalho [CARECE DE FONTES]. % Fred Brooks: The Mythical Man-Month

[FALAR DAS LIMITAÇÕES DAS REDES SOCIAIS DE SOBRE COMO COMUNICAÇÃO FACE A FACE É MELHOR].
% 4 - Distance Matters (Olson and Olson, 2000)

[FALAR DOS CUSTOS DA COMUNICACAO FACE A FACE E SE SUAS LIMITACOES]
% 4 - Distance Matters (Olson and Olson, 2000)

Nota-se então a importância do fator pessoal para o aprendizado dos desenvolvedores de software modernos.

\section{Rotatividade de pessoal}

Rotatividade de pessoal é inerente a qualquer área de atuação da indústria [CARECE DE FONTES].
Sua ocorrência acarreta em [PROBLEMAS].
% staff turnover (software development), Employee turnover, turnover open source development

Em equipes de desenvolvedores de software, a partida de um membro pode acarretar em [FALAR DE COISAS COMO PERDA DE INFORMAÇÕES] caso não haja nenhuma forma de registro em mídia da mesma.
% knowledge management, tacit knowledge, software development (organization or team)

Em geral, novos membros da equipe passam por um período de adaptação onde devem aprender padrões, práticas e ferramentas utilizadas pelo grupo [CARECE DE FONTES].
% 5 - PEDIR A FERNANDO
% Knowledge management and organizational culture in a software organization: a case study
% https://scholar.google.com/citations?view_op=view_citation&hl=en&user=I8o8rfoAAAAJ&sortby=pubdate&citation_for_view=I8o8rfoAAAAJ:maZDTaKrznsC


% 6 - Social Barriers Faced by Newcomers https://scholar.google.com/citations?view_op=view_citation&hl=en&user=I8o8rfoAAAAJ&sortby=pubdate&citation_for_view=I8o8rfoAAAAJ:r0BpntZqJG4C

Em casos específicos, como desenvolvedores iniciantes (em geral chamados de \"desenvolvedores júnior\", no mercado), ainda existe a necessidade em se adquirir conhecimento sobre as tecnologias utlizadas pela equipe ao qual foi recém integrado.

Todo esse conhecimento é em geral transmitido oralmente [CARECE DE FONTES] ou por mídias que não oferecem um bom suporte à sua manutenção [CARECE DE FONTES].
% neste parágrafo você explicita quais são os problemas das abordagens atuais. ao mesmo tempo, você justifica o seu trabalho.

Sendo assim, uma ferramenta que estimule seus usuários a registrarem informações fundamentais para a equipe tende a evitar tanto que informações sejam perdidas no caso de partidas quanto facilitar o aprendizado e adaptação de novos membros à equipe.

\section{Gestão e recuperação de informação}

A gestão de informações é a área [ARRUMAR DEFINIÇÃO].
% 5 - PEDIR A FERNANDO
% Knowledge management and organizational culture in a software organization: a case study


Uma boa gestão de informação e facilidade de se recuperar informações permitem que [ARRUMAR BENEFÍCIOS].

É possível ver então que uma equipe de desenvolvimento que apresente alguma forma de gestão de informações pode garantir facilitar e estimular, por exemplo, o reuso de soluções já conhecidas de problemas recorrentes.

% \section{GitHub}

% 7 - A study on the geographical distribution

\section{Reuso em engenharia de software}

Reuso é uma das caracterísitcas chaves de um código de qualidade. Sua definição é [ARRUMAR DEFINIÇÃO DE REUSO].
% ver o livro da Gang of Four

Bons engenheiros de software procuram desenvolver código com boa possibilidade de reuso, caso seja conveniente [CARECE DE FONTES].

É notável então que uma boa forma de garantir a gerência de informações visando o reuso de conteúdo produzido por uma equipe pode trazer melhorias para a qualidade do código produzido pela mesma.

% 1
% @inproceedings{storey2014r,
%   title={The (R) Evolution of social media in software engineering},
%   author={Storey, Margaret-Anne and Singer, Leif and Cleary, Brendan and Figueira Filho, Fernando and Zagalsky, Alexey},
%   booktitle={Proceedings of the on Future of Software Engineering},
%   pages={100--116},
%   year={2014},
%   organization={ACM}
% }
%

% 2
% @inproceedings{singer2014software,
%   title={Software engineering at the speed of light: How developers stay current using twitter},
%   author={Singer, Leif and Figueira Filho, Fernando and Storey, Margaret-Anne},
%   booktitle={Proceedings of the 36th International Conference on Software Engineering},
%   pages={211--221},
%   year={2014},
%   organization={ACM}
% }


% 3
% @article{treude2012programming,
%   title={Programming in a socially networked world: the evolution of the social programmer},
%   author={Treude, Christoph and Figueira Filho, Fernando and Cleary, Brendan and Storey, Margaret-Anne},
%   journal={The Future of Collaborative Software Development},
%   pages={1--3},
%   year={2012}
% }

% 4
% article{Olson:2000:DM:1463015.1463019,
%  author = {Olson, Gary M. and Olson, Judith S.},
%  title = {Distance Matters},
%  journal = {Hum.-Comput. Interact.},
%  issue_date = {September 2000},
%  volume = {15},
%  number = {2},
%  month = sep,
%  year = {2000},
%  issn = {0737-0024},
%  pages = {139--178},
%  numpages = {40},
%  url = {http://dx.doi.org/10.1207/S15327051HCI1523_4},
%  doi = {10.1207/S15327051HCI1523_4},
%  acmid = {1463019},
%  publisher = {L. Erlbaum Associates Inc.},
%  address = {Hillsdale, NJ, USA},
% }

% 5
% @inproceedings{rabelo2015knowledge,
%   title={Knowledge management and organizational culture in a software organization: a case study},
%   author={Rabelo, Jacilane and Oliveira, Edson and Viana, Davi and Braga, Lu{\'\i}s and Santos, Gleison and Steinmacher, Igor and Conte, Tayana},
%   booktitle={Proceedings of the Eighth International Workshop on Cooperative and Human Aspects of Software Engineering},
%   pages={89--92},
%   year={2015},
%   organization={IEEE Press}
% }

% 6
% @inproceedings{steinmacher2015social,
%   title={Social barriers faced by newcomers placing their first contribution in open source software projects},
%   author={Steinmacher, Igor and Conte, Tayana and Gerosa, Marco Aur{\'e}lio and Redmiles, David},
%   booktitle={Proceedings of the 18th ACM conference on Computer supported cooperative work \& social computing},
%   pages={1379--1392},
%   year={2015},
%   organization={ACM}
% }

% 7
% @article{figueira2015study,
%   title={A study on the geographical distribution of Brazil’s prestigious software developers},
%   author={Figueira Filho, Fernando and Perin, Marcelo Gattermann and Treude, Christoph and Marczak, Sabrina and Melo, Leandro and da Silva, Igor Marques and dos Santos, Lucas Bibiano},
%   journal={Journal of Internet Services and Applications},
%   volume={6},
%   number={1},
%   pages={1--12},
%   year={2015},
%   publisher={Springer London}
% }
