\chapter{Introdução}

Desenvolvedores de software são trabalhadores do conhecimento e frequentemente precisam reutilizar o conhecimento aprendido por outros desenvolvedores, de modo a proporcionar um melhor desempenho para toda organização~\cite{Druker1993}~\cite{Wiig2003}.
É esperado que os trabalhadores do conhecimento melhorem  continuamente seu trabalho em um processo em culmina na melhora significativa da sua empresa~\cite{Kavitha2011}.

% IM: Fernando, posso passar esses dois parágrafos abaixo pra revisão?

Em equipes de desenvolvimento de software, o reuso de conhecimento ocorre utilizando uma variedade de artefatos, como código fonte, requisitos, modelos, dados e padrões~\cite{Levy2009}, bem como através de interações face a face, comunicação escrita e também via o repasse de referências de documentação, links, dentre outros~\cite{Storey2014}~\cite{Olson2000}~\cite{CubraniC2004}.

% FF: o parágrafo abaixo ainda não está muito claro. Se não estiver seguro ainda para falar um pouco mais sobre reuso de conhecimento, sugiro remover todo o parágrafo e ir direto pro parágrafo abaixo... A troca de informações...
% IM: Removi a sentença que achei confusa de ser traduzida. Creio que fluiu melhor agora.
Neste quesito, entra a gerência de conhecimento, encarregada da elicitação, armazenamento, gerenciamento e reuso do conhecimento em suas diferentes formas.
\ignore{FONTE 2}.
Dentre esse reuso, se encontra, por exemplo, como se deu a implementação de uma determinada funcionalidade em um determinado projeto de software.

% FONTE 2 - KM is comprised of the elicitation, packaging and management, and reuse of knowledge in all of its different forms, and in particular, software engineering artifacts as code, design, requirements, models, data, standards, and lessons learned.
%Software development can be improved by recognizing the related knowledge content and structure as well as the required knowledge, and performing planning activities. - 05071412.pdf

% IM: Fernando, não consigo achar esse artigo da FONTE 2, vc poderia me passar por favor?
% IM: Seria esse? http://www.computer.org/csdl/proceedings/chase/2009/3712/00/05071412-abs.html

A troca de informações é um mecanismo fundamental para que o reuso de conhecimento ocorra de maneira eficiente em equipes de desenvolvimento. Desenvolvedores em fase de aprendizado de uma determinada tecnologia comumente passam por situações e problemas semelhantes. Quando se deparam com algum tipo de adversidade, procuram por alguma fonte de informação capaz de auxiliá-los por tal problema. Existe então o desperdício de tempo de se recuperar tal informação (muitas vezes, interferindo nas atividades de um colega de trabalho) para se resolver um problema que já é de conhecimento da equipe dado alguma experiência anterior.
% FF: neste parágrafo seria melhor dar um exemplo concreto mesmo, mas por enquanto deixa assim.
% IM: posso dar esse exemplo abstrato acima e um concreto abaixo?

% IM: Fernando, posso botar o parágrafo abaixo para a revisão?

Com a falta de uma boa gerência de conhecimento, é comum desenvolvedores implementarem funcionalidades semelhantes em diferentes contextos usando abordagens ad-hoc~\cite{SangMok2011}. Essas novas implementações, principalmente se feitas por desenvolvedores menos experientes, tendem a ser menos otimizadas e elegantes que soluções já previamente implementadas pela equipe, podendo gerar um débito técnico a longo prazo.
% FF: A sentença acima não está clara.
% IM: Adicionei uma frase depois para ver se fica mais clara.

Desenvolvedores mais experientes em um determinado projeto tendem a atuar como mentores~\cite{CubraniC2004} e tal ato, como atividade de gerência de conhecimento, tende a consumir recursos~\cite{Wiig2003}. Um dos pilares da gerência de conhecimento é a melhoria da produtividade através do compartilhamento e transferência eficiente de conhecimento, que tende a consumir tempo e as vezes até impossível de se realizar~\cite{Levy2009}.
Muitas das interações entre desenvolvedores são informais e não há registro que possa ser consultado para propocionar o reuso do conhecimento trocado~\cite{Olson2000}. Um agravante tipicamente é a frequente rotatividade de membros em equipes. Nesses casos, a saída de um membro da equipe que detém determinado conhecimento pode prejudicar toda organização. De fato, a falta de uma abordagem mais sistemática para proporcionar o reuso de conhecimento pode estar associada ao fracasso de projetos em organizações~\cite{Hall2008}.

Assim, a elaboração de uma ferramenta capaz de agregar referências de código e tarefas a soluções pode trazer enormes benefícios a equipes de desenvolvimento~\cite{CubraniC2004}. Este estudo propôs a elaboração de tal ferramenta. Esta atua como um catálogo, agregando referências de código, informações externas e comentários fornecidos por desenvolvedores da equipe com o intuito de auxiliar outros desenvolvedores a buscar em fontes da própria equipe como se deram implementações de funcionalidades já feitas em outros projetos.

Segue um exemplo de uso da ferramenta: Um determinado desenvolvedor X, experiente dentro de sua equipe, percebe que muitos membros novatos procuram sua ajuda para implementar uma funcionalidade de exportação de uma página web para um documento em formato PDF. X então pode fazer uso da ferramenta, para registrar como se realiza a implementação de tal funcionalidade solicitada. Ele informa um título, descrição curta (para facilitar futuras buscas) e uma descrição de como se dá tal implementação. Como ilustração em formato de código, o desenvolvedor pode utilizar de links de trechos de código disponívels artefatos de código já implementados pela sua equipe. Esses links são renderizados no editor de texto, sem a necessidade de X copiar e colar o código em si dentro do editor. X também pode complementar sua descrição adicionando links para outras páginas web (respostas de sites como Stack Overflow, outros tutoriais, etc...) em forma de anexo da documentação. Por fim, X informa tags relevantes ao artefato que está produzindo, facilitando a sua recuperação por parte de outros membros de sua equipe.

Outros desenvolvedores, a partir de então, podem recorrer diretamente a ferramenta quando precisarem implementar a funcionalidade de exportação de página em formato PDF. Dessa forma, X poderá se dedicar mais a outras atividades de seu dia-a-dia de trabalho.

% IM: Posso diminuir a explicação sobre a 4Soft (Como o lance de ser formada só por alunos de BES e BTI) e passar pra parte do Método?

Este estudo descreve a elaboração de tal ferramenta e sua avaliação em dois contextos. O primeiro é o de uma equipe real de desenvolvimento de software e o segundo contexto é o de alunos de graduação da área de computação enquanto cursavam a disciplina de Desenvolvimento Colaborativo de Software. A ferramenta foi utilizada para se auxiliarem durante o desenvolvimento do projeto final de disciplina.

O objetivo geral deste trabalho é responder as seguintes perguntas:

\begin{enumerate}
  \item A ferramenta diminui o esforço gasto com mentoria de desenvolvedores?
  \item Desenvolvedores preferem utilizar a ferramenta aos métodos tradicionais de aprendizado/ resolução de dúvidas?
  \item Quais atributos da ferramenta oferecem vantagens da mesma em relação aos métodos tradicionais de aprendizado/ resolução de dúvidas?
\end{enumerate}

% IM: To falando muito "A ferramenta"...

% O estudo também realiza uma análise crítica dos impactos do uso da mesma no contexto de empresa. Dentre os impactos previstos estão a redução da necessidade de mentoria ou consulta para os casos de  reuso de funcionalidades em diferentes contextos. Tais impactos visam contribuir para a redução de gastos de recursos relacionados a tempo de implementação de funcionalidades, esforço de recuperação de informações, orientação de colaboradores e trabalho dos desenvolvedores que atuam como mentores dentro da equipe.
% IM: Vou tirar esse parágrafo (acima) por ter botado as perguntas de pesquisa

Para responder as perguntas de pesquisa propostas, foi feito aplicado um questionário com usuários de ferramenta, além de entrevistas semi-estruturadas com alguns dos usuários.

Concluiu-se que...
% IM: Vou precisar dizer os resultados aqui, né, professor?

O Capítulo 2 cita as referências utilizadas neste estudo como o uso de mídias pelo engenheiro de software, os efeitos da rotatividade em equipes de software, além de gestão e recuperação de informação. O Capítulo 3 explica em detalhes os procedimentos tomados para a construção e análise da ferramenta. O Capítulo 6 descreve em detalhes a ferramenta elaborada. O Capítulo 5 realiza a análise dos resultados dos experimentos bem como discute seus resultados. O Capítulo 6 aborda as considerações finais sobre o trabalho.
% IM: tá muito "crua"?

% FF: algumas coisas para você incluir aqui:
% 1- como será feita esta análise? que métodos de avaliação serão utilizados? (colocar de forma bem resumida)
% IM: Gostaria de conversar sobre este ponto com vc depois.

% 3- de onde veio o texto abaixo em inglês? ele está melhor que o em português acima.
% IM: Acredito que o texto veio de algum slide que fizemos. Mas coloquei uma descrição de um caso de uso acima. Fiz também uma descrição geral da ferramenta na linha 37. Está vaga demais?

% In software development, it is common for a member of the team to implement a feature very similar to one that was already done by a teammate. In some cases, that is a cumbersome task. Searching through documentation and code, as well as asking other teammates can be time and resource demanding.
% We propose a tool that will permit a team to store documentation for features via GitHub issues and comm its merged with other information like Stack Overflow questions, framework or language documentation or even notes written by teammates. This tool will be designed, tested and evaluated by the developers of 4Soft, the Junior Enterprise of Software Engineer of UFRN.

% FF: coloque um .gitignore pra LaTeX e arquivos relacionados? :)
% IM: Não entendi :/. Se eu colocar um gitignore pra arquivos relacionados não vamos ter como compartilhar um com o outro via git. Além disso, o overleaf não permite o upload de PDFs dentro da estrutura dos documentos, por exemplo. Posso compartilhar uma pasta de referencias minhas no dropbox com vc, que tal?