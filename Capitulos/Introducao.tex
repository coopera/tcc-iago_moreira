\chapter{Introdução}

No contexto de trabalhadores do conhecimento, como na área de desenvolvimento de software, é diária a troca de conhecimento com o objetivo de se obter um melhor desempenho para a organização inteira~\cite{Druker1993, Wiig2003}. A gerência de conhecimento lida com o reuso de conhecimento em  suas diferentes formas, como: design de código, requisitos, modelos, dados, padrões e lições aprendidas~\cite{Levy2009}. Um dos casos de reuso é o conhecimento de como se deu a implementação de uma determinada funcionalidade em um determinado projeto de software. Exemplo: diferentes projetos podem implementar cadastro de usuário via Google ou Facebook, cada um com certas especificidades.

É comum também que essa implementação em diferentes contextos seja feita através de abordagens \textit{ad-hoc} (reimplementação completa de trechos de código com poucas modificações e sem modularizacão visando reuso)~\cite{SangMok2011}. Nesse caso, desenvolvedores mais experientes em um determinado projeto ou que já implementou tal funcionalidade tendem a atuar como mentores~\cite{CubraniC2004} e tal ato, como atividade de gerência de conhecimento, acarreta no despendimento de recursos, principalmente dos mentores~\cite{Wiig2003}.

Atualmente, as maneiras mais comuns de troca de informações entre desenvolvedores (incluindo, obviamente, a mentoria) são via oral, escrita ou repasse de referências (documentação, links externos, etc)~\cite{Storey2014, Olson2000, CubraniC2004}. Em alguns casos, pela própria natureza do meio, comunicação não se mantém registrada física ou virtualmente. Um dos exemplos de comunicação sem registro é a comunicação exclusivamente oral em que, ao fim do diálogo, as informações trocadas ficam apenas na memória dos envolvidos, sem maneiras de terceiros consultarem posteriormente aquelas informações~\cite{Olson2000}.

Além disso, existe o fator da rotatividade de membros em equipes, sempre associado a custos de transferência de conhecimento e treinamento~\cite{Hall2008}, possuindo relação com sucesso ou fracasso de projetos de software~\cite{Hall2008}.

Assim, a elaboração de uma ferramenta capaz de agregar referências de código e tarefas a soluções pode trazer enormes benefícios a equipes de desenvolvimento~\cite{CubraniC2004}. A ferramenta atua como um catálogo, agregando referências de código, informações externas e comentários fornecidos por desenvolvedores da equipe com o intuito de auxiliar outros desenvolvedores a buscar em fontes da própria equipe como se deram implementações de funcionalidades semelhantes às que já foram feitas em outros projetos.

Segue um exemplo de uso da ferramenta: Um determinado desenvolvedor X, experiente dentro de sua equipe, percebe que muitos membros novatos procuram sua ajuda para implementar uma funcionalidade de exportação de uma página web para um documento em formato PDF. X então pode fazer uso da ferramenta, para registrar como se realiza a implementação de tal funcionalidade solicitada. Ele informa um título, descrição curta (para facilitar futuras buscas) e uma descrição de como se dá tal implementação. Como ilustração em formato de código, o desenvolvedor pode utilizar de links de trechos de código disponívels artefatos de código já implementados pela sua equipe. Esses links são renderizados no editor de texto, sem a necessidade de X copiar e colar o código em si dentro do editor. X também pode complementar sua descrição adicionando links para outras páginas web (respostas de sites como Stack Overflow, outros tutoriais, etc...) em forma de anexo da documentação. Por fim, X informa tags relevantes ao artefato que está produzindo, facilitando a sua recuperação por parte de outros membros de sua equipe.

Outros desenolvedores, a partir de então, podem recorrer diretamente a ferramenta quando precisarem implementar a funcionalidade de exportação de página em formato PDF. Dessa forma, X poderá se dedicar mais a outras atividades de seu dia-a-dia de trabalho.

Este estudo descreve a elaboração de tal ferramenta com a participação de uma equipe real de desenvolvimento de software. A equipe em questão é a 4Soft\footnote{\url{http://www.4softjr.com.br}}, empresa júnior\footnote{\url{http://en.wikipedia.org/wiki/Junior_enterprise}} dos cursos de Engenharia de Software e Tecnologia da Informação da Universidade Federal do Rio Grande do Norte (UFRN\footnote{\url{http://www.ufrn.br}}). A empresa atua na área de desenvolvimento de software Web para clientes de diversos ramos e é formada exclusivamente por alunos dos cursos de Bacharelado em Engenharia de Software\footnote{\url{http://www.dimap.ufrn.br/pt/graduacao/engenharia-de-software/apresentacao}} e Bacharelado em Tecnologia da Informação\footnote{\url{http://www.imd.ufrn.br/curso_bacharelado.php}} da UFRN.

O estudo também realiza uma análise crítica dos impactos do uso da mesma no contexto de empresa. Dentre os impactos previstos estão a redução da necessidade de mentoria ou consulta para os casos de  reuso de funcionalidades em diferentes contextos. Tais impactos visam contribuir para a redução de gastos de recursos relacionados a tempo de implementação de funcionalidades, esforço de recuperação de informações, orientação de colaboradores e trabalho dos desenvolvedores que atuam como mentores dentro da equipe.

% falta adaptar para a estrutura do TCC

% O Capítulo 2 descreve em detalhes o objetivo pretendido pelo trabalho e a quais perguntas pretende-se responder. Em seguida, o Capítulo 3 aborda a metodologia utilizada para a execução do estudo e que atividades serão desempenhadas pelos envolvidos. O Capítulo 4, por fim, descreve o cronograma das atividades descritas no capítulo anterior.
