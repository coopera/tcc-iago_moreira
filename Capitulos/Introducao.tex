\chapter{Introdução}

%ORIGINAL
% No contexto de trabalhadores do conhecimento, como na área de desenvolvimento de software, é diária a troca de conhecimento com o objetivo de se obter um melhor desempenho para a organização inteira~\cite{Druker1993, Wiig2003}. A gerência de conhecimento lida com o reuso de conhecimento em  suas diferentes formas, como: design de código, requisitos, modelos, dados, padrões e lições aprendidas~\cite{Levy2009}. Um dos casos de reuso é o conhecimento de como se deu a implementação de uma determinada funcionalidade em um determinado projeto de software. Exemplo: diferentes projetos podem implementar cadastro de usuário via Google ou Facebook, cada um com certas especificidades.


%FERNANDO 25/08
Desenvolvedores de software são trabalhadores do conhecimento e frequentemente precisam reutilizar o conhecimento aprendido por outros desenvolvedores, de modo a proporcionar um melhor desempenho para toda organização~\cite{Druker1993}~\cite{Wiig2003}.\ignore{FONTE 1 - COMEÇO}
É previsto que os trabalhadores do conhecimento melhorem  continuamente seu trabalho em um processo em culmina na melhora significativa da sua empresa. [PEGAR FONTE 6][PEGAR FONTE 21]\ignore{FONTE 1 - FIM}
% IM: Mudei a frase acima. Melhor agora? :)
Em equipes de desenvolvimento de software, o reuso de conhecimento ocorre utilizando uma variedade de artefatos, como código fonte, requisitos, modelos, dados e padrões~\cite{Levy2009}\ignore{FONTE 2 - COMEÇO}, bem como através de interações face a face, comunicação escrita e também via o repasse de referências de documentação, links, dentre outros~\cite{Storey2014}~\cite{Olson2000}~\cite{CubraniC2004}\ignore{FONTE 3}

% FF: o parágrafo abaixo ainda não está muito claro. Se não estiver seguro ainda para falar um pouco mais sobre reuso de conhecimento, sugiro remover todo o parágrafo e ir direto pro parágrafo abaixo... A troca de informações...
% IM: Removi a sentença que achei confusa de ser traduzida. Creio que fluiu melhor agora.
Neste quesito, entra a gerência de conhecimento, encarregada da elicitação, armazenamento, gerenciamento e reuso do conhecimento em suas diferentes formas.
\ignore{FONTE 2 - FIM}.
Dentre esse reuso, se encontra, por exemplo, como se deu a implementação de uma determinada funcionalidade em um determinado projeto de software.

% FONTE 1 - Knowledge workers are required to improve their work on a daily basis in a process that cumulates into a significant improvement in performance for the entire enterprise [6][21].
% IM - TRADUÇÃO:
% É previsto que os trabalhadores do conhecimento melhorem  continuamente seu trabalho em um processo em culmina na melhora significativa da sua empresa como um todo

% FONTE 2 - KM is comprised of the elicitation, packaging and management, and reuse of knowledge in all of its different forms, and in particular, software engineering artifacts as code, design, requirements, models, data, standards, and lessons learned.
%Software development can be improved by recognizing the related knowledge content and structure as well as the required knowledge, and performing planning activities. - 05071412.pdf
% IM - TRADUÇÃO:
% A gerência de conhecimento está encarregada da elicitação, armazenamento e gerenciamento e reuso do conhecimento em suas diferentes formas, em particular, artefatos da engenharia de software como código, arquitetura, requisitos, modelos, dados, padrões e lições aprendidas.
% Desenvolvimento de software pode ser melhorando se reconhecendo o conteúdo e estrutura do conhecimento relacionado bem como de conhecimento necessário, além de atividades de planejamento.

%-------

% ORIGINAL
% É comum também que essa implementação em diferentes contextos seja feita através de abordagens \textit{ad-hoc} (reimplementação completa de trechos de código com poucas modificações e sem modularizacão visando reuso)~\cite{SangMok2011}. Nesse caso, desenvolvedores mais experientes em um determinado projeto ou que já implementou tal funcionalidade tendem a atuar como mentores~\cite{CubraniC2004} e tal ato, como atividade de gerência de conhecimento, acarreta no despendimento de recursos, principalmente dos mentores~\cite{Wiig2003}.
%
% Atualmente, as maneiras mais comuns de troca de informações entre desenvolvedores (incluindo, obviamente, a mentoria) são via oral, escrita ou repasse de referências (documentação, links externos, etc)~\cite{Storey2014, Olson2000, CubraniC2004}. Em alguns casos, pela própria natureza do meio, comunicação não se mantém registrada física ou virtualmente. Um dos exemplos de comunicação sem registro é a comunicação exclusivamente oral em que, ao fim do diálogo, as informações trocadas ficam apenas na memória dos envolvidos, sem maneiras de terceiros consultarem posteriormente aquelas informações~\cite{Olson2000}.

A troca de informações é um mecanismo fundamental para que o reuso de conhecimento ocorra de maneira eficiente em equipes de desenvolvimento. Desenvolvedores em fase de aprendizado de uma determinada tecnologia comumente passam por situações e problemas semelhantes. Quando se deparam com algum tipo de adversidade, procuram por alguma fonte de informação capaz de auxiliá-los por tal problema. Existe então o desperdício de tempo de se recuperar tal informação (muitas vezes, interferindo nas atividades de um colega de trabalho) para se resolver um problema que já é de conhecimento da equipe dado alguma experiência anterior.
% FF: neste parágrafo seria melhor dar um exemplo concreto mesmo, mas por enquanto deixa assim.
% IM: posso dar esse exemplo abstrato acima e um concreto abaixo?

Com a falta de uma boa gerência de conhecimento, é comum desenvolvedores implementarem funcionalidades semelhantes em diferentes contextos usando abordagens ad-hoc~\cite{SangMok2011}. Essas novas implementações, principalmente se feitas por desenvolvedores menos experientes, tendem a ser menos otimizadas e elegantes que soluções já previamente implementadas pela equipe, podendo gerar um débito técnico a longo prazo.
% FF: A sentença acima não está clara.
% IM: Adicionei uma frase depois para ver se fica mais clara. Procurei não mexer na frase em si por ser uma citação.

Desenvolvedores mais experientes em um determinado projeto tendem a atuar como mentores~\cite{CubraniC2004} e tal ato, como atividade de gerência de conhecimento, tende a consumir recursos~\cite{Wiig2003}.\ignore{FONTE 4 - COMEÇO} Um dos pilares da gerência de conhecimento é a melhoria da produtividade através do compartilhamento e transferência eficiente de conhecimento, que tende a consumir tempo e as vezes até impossível de se realizar. \ignore{FONTE 4 - FIM}
Muitas das interações entre desenvolvedores são informais e não há registro que possa ser consultado para propocionar o reuso do conhecimento trocado~\cite{Olson2000}. Um agravante tipicamente é a frequente rotatividade de membros em equipes. Nesses casos, a saída de um membro da equipe que detém determinado conhecimento pode prejudicar toda organização. De fato, a falta de uma abordagem mais sistemática para proporcionar o reuso de conhecimento pode estar associada ao fracasso de projetos em organizações~\cite{Hall2008}.
% FONTE 4 - One of the cornerstones of KM is improving productivity by effective sharing and transfer of knowledge, which tends to be time-consuming and often impossible [21] - 05071412.pdf.
% IM- TRADUÇÃO:
% Um dos pilares da gerência de conhecimento é a melhoria da produtividade através do compartilhamento e transferência eficiente de conhecimento, que tende a consumir tempo e as vezes até impossível de se realizar.

%-------

Assim, a elaboração de uma ferramenta capaz de agregar referências de código e tarefas a soluções pode trazer enormes benefícios a equipes de desenvolvimento~\cite{CubraniC2004}. A ferramenta atua como um catálogo, agregando referências de código, informações externas e comentários fornecidos por desenvolvedores da equipe com o intuito de auxiliar outros desenvolvedores a buscar em fontes da própria equipe como se deram implementações de funcionalidades já feitas em outros projetos.

Segue um exemplo de uso da ferramenta: Um determinado desenvolvedor X, experiente dentro de sua equipe, percebe que muitos membros novatos procuram sua ajuda para implementar uma funcionalidade de exportação de uma página web para um documento em formato PDF. X então pode fazer uso da ferramenta, para registrar como se realiza a implementação de tal funcionalidade solicitada. Ele informa um título, descrição curta (para facilitar futuras buscas) e uma descrição de como se dá tal implementação. Como ilustração em formato de código, o desenvolvedor pode utilizar de links de trechos de código disponívels artefatos de código já implementados pela sua equipe. Esses links são renderizados no editor de texto, sem a necessidade de X copiar e colar o código em si dentro do editor. X também pode complementar sua descrição adicionando links para outras páginas web (respostas de sites como Stack Overflow, outros tutoriais, etc...) em forma de anexo da documentação. Por fim, X informa tags relevantes ao artefato que está produzindo, facilitando a sua recuperação por parte de outros membros de sua equipe.

Outros desenolvedores, a partir de então, podem recorrer diretamente a ferramenta quando precisarem implementar a funcionalidade de exportação de página em formato PDF. Dessa forma, X poderá se dedicar mais a outras atividades de seu dia-a-dia de trabalho.

Este estudo descreve a elaboração de tal ferramenta com a participação de uma equipe real de desenvolvimento de software. A equipe em questão é a 4Soft\footnote{\url{http://www.4softjr.com.br}}, empresa júnior\footnote{\url{http://en.wikipedia.org/wiki/Junior_enterprise}} dos cursos de Engenharia de Software e Tecnologia da Informação da Universidade Federal do Rio Grande do Norte (UFRN\footnote{\url{http://www.ufrn.br}}). A empresa atua na área de desenvolvimento de software Web para clientes de diversos ramos e é formada exclusivamente por alunos dos cursos de Bacharelado em Engenharia de Software\footnote{\url{http://www.dimap.ufrn.br/pt/graduacao/engenharia-de-software/apresentacao}} e Bacharelado em Tecnologia da Informação\footnote{\url{http://www.imd.ufrn.br/curso_bacharelado.php}} da UFRN.

O estudo também realiza uma análise crítica dos impactos do uso da mesma no contexto de empresa. Dentre os impactos previstos estão a redução da necessidade de mentoria ou consulta para os casos de  reuso de funcionalidades em diferentes contextos. Tais impactos visam contribuir para a redução de gastos de recursos relacionados a tempo de implementação de funcionalidades, esforço de recuperação de informações, orientação de colaboradores e trabalho dos desenvolvedores que atuam como mentores dentro da equipe.

% FF: algumas coisas para você incluir aqui:
% 1- como será feita esta análise? que métodos de avaliação serão utilizados? (colocar de forma bem resumida)
% 2- como está organizada a monografia?
% 3- de onde veio o texto abaixo em inglês? ele está melhor que o em português acima.
% FF: coloque um .gitignore pra LaTeX e arquivos relacionados? :)
% In software development, it is common for a member of the team to implement a feature very similar to one that was already done by a teammate. In some cases, that is a cumbersome task. Searching through documentation and code, as well as asking other teammates can be time and resource demanding.
% We propose a tool that will permit a team to store documentation for features via GitHub issues and comm its merged with other information like Stack Overflow questions, framework or language documentation or even notes written by teammates. This tool will be designed, tested and evaluated by the developers of 4Soft, the Junior Enterprise of Software Engineer of UFRN.

% falta adaptar para a estrutura do TCC

% O Capítulo 2 descreve em detalhes o objetivo pretendido pelo trabalho e a quais perguntas pretende-se responder. Em seguida, o Capítulo 3 aborda a metodologia utilizada para a execução do estudo e que atividades serão desempenhadas pelos envolvidos. O Capítulo 4, por fim, descreve o cronograma das atividades descritas no capítulo anterior.
