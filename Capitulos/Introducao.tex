\chapter{Introdução}

Processos e metodologias de desenvolvimento estão entre os principais temas da pesquisa em Engenharia de Software [carece de fontes]. Um processo, quando executado de forma correta, tem o impacto na melhoria da qualidade do software produzido e na redução de custos de projeto. Nesse contexto, um dos desafios de empresas de software é o de motivar membros do time de desenvolvimento a aderirem a boas práticas, como a produção adequada de testes, a melhor comunicação entre membros do time e o comprometimento com metas de produção [2].

A gamificação é o uso de elementos de design de jogos em contextos que não sejam jogos [1], e tem sido usada com sucesso em uma variedade de aplicações, dentre elas, sites da Web cujo público-alvo são os desenvolvedores de software. Por exemplo, o site Stack Overflow facilita a troca de conhecimento entre os desenvolvedores de software ao oferecer um serviço que permite com que usuários postem perguntas que são respondidas por outros usuários. O site usa gamificação para incentivar e premiar a participação da comunidade. Por exemplo, usuários recebem pontos por enviarem respostas úteis, eventualmente ganhando insígnias, ou badges, pelos seus serviços e contribuições à comunidade. Em sites como Coderwall e Coderbits, usuários ganham badges por conquistas no âmbito do desenvolvimento de software livre, o que estimula o aprendizado de novas tecnologias entre os desenvolvedores de software (Singer 2013). Este trabalho propõe o uso de mecânicas de jogos, ou gamificação, com o intuito de melhorar processos de desenvolvimento de software em empresas e times de desenvolvimento. 

\section{Exemplo de Uso da Ferramenta}

Um determinado desenvolvedor João, experiente dentro de sua equipe, percebe que muitos membros novatos procuram sua ajuda para implementar uma funcionalidade de exportação de uma página web para um documento em formato PDF. João então pode fazer uso da ferramenta para registrar como se realiza a implementação de tal funcionalidade solicitada. Ele informa um título, descrição curta (para facilitar futuras buscas) e uma descrição de como se dá tal implementação. Como exemplificação em formato de código, o desenvolvedor pode utilizar de links de trechos de código disponíveis artefatos de código já implementados pela sua equipe. Esses links são renderizados no editor de texto, sem a necessidade de João copiar e colar o código em si dentro do editor. João também pode complementar sua descrição adicionando links para outras páginas web (respostas de sites como Stack Overflow, outros tutoriais, etc...) em forma de anexo da documentação. Por fim, João informa \textit{tags} relevantes ao artefato que está produzindo, facilitando a sua recuperação por parte de outros membros de sua equipe.

Outros desenvolvedores, a partir de então, podem recorrer diretamente a ferramenta quando precisarem implementar a funcionalidade de exportação de página em formato PDF. Dessa forma, João poderá se dedicar mais a outras atividades de seu dia-a-dia de trabalho.

\section{Objetivos e Perguntas de Pesquisa}

Este trabalho tem por objetivo principal analisar e compreender como a gamificação pode contribuir para melhoria de processos de software em empresas e times de desenvolvimento corporativos, visando responder às seguintes perguntas de pesquisa:

\begin{enumerate}
  \item Como projetar mecânicas de jogos que contribuam para melhoria de processos de desenvolvimento de software?
  \item Como mecânicas de jogos podem encorajar desenvolvedores em empresas de software a aderirem a boas práticas?
  \item Qual é o processo para construção, teste e implantação de mecânicas de jogos em empresas de software?
  \item Quais são os desafios e limitações do uso de mecânicas de jogos em empresas de software?
\end{enumerate}

\section{Estrutura do Trabalho}

O Capítulo 2 apresenta a fundamentação teórica relacionada a este trabalho, focando em mineração de repositórios de software, os conceitos de \textit{awareness} em equipes de software, além de gestão e recuperação de informação. O Capítulo 3 explica em detalhes os procedimentos tomados para a construção e análise da ferramenta tendo como base os conhecimentos citados no capítulo anterior. O Capítulo 4 descreve em detalhes a ferramenta elaborada. O Capítulo 5 realiza a análise dos resultados dos experimentos bem como discute seus resultados. O Capítulo 6 aborda as considerações finais sobre o trabalho.
