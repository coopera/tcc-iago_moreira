\chapter{Projeto de Solução}

Com base nos requisitos então elicitados, pôde-se ter uma noção clara de como seria feita a implementação da aplicação e este processo teve início. Ao longo dele, feedback a respeito da estética da ferramenta foi constantemente colhido com a equipe de desenvolvedores da empresa júnior bem como de parceiros do grupo de pesquisa do qual o autor faz parte.

\section{Listagem de Documentos}

Na tela inicial da aplicação é possível ver uma lista com todos os documentos já criados. Cada documento é exibido com seu título, descrição, lista de \textit{tags} e autor (caso tenha se identificado durante a criação do documento).

https://www.dropbox.com/s/4rput1kdwxxiknj/Screenshot%202015-11-05%2016.08.17.png?dl=0

Ao fim da lista de documentos, é possível ver quantos documentos se encontram nela e também uma lista das \textit{tags} presentes, incluindo sua quantidade de ocorrências dentro do grupo de documentos exibidos.

https://www.dropbox.com/s/bdhexdhgwrgo6fl/Screenshot%202015-11-05%2016.08.28.png?dl=0

ou

https://www.dropbox.com/s/us93gna8m8k4hkb/Screenshot%202015-11-05%2016.08.48.png?dl=0

ou

https://www.dropbox.com/s/3wm1tq2s6gqsrnq/Screenshot%202015-11-05%2016.08.59.png?dl=0

\section{Criação de um Documento}

\subsection{Campos}
\subsection{Implementação}

\begin{itemize}
  \item Importação de Arquivo
  \item Importação de Linha Arquivo
  \item Importação de Intervalos de Linha Arquivo
  \item Importação de Commit
  \item Importação de Pull Request
\end{itemize}

\subsection{Markdown}
\subsection{Exibição do Documento}

\section{Recuperação de Documentos}

\subsection{Através de Tags}
\subsection{Através de Atributos}
% falar do "your twydies"

\section{Login com GitHub}

Para que os usuários possam importar informações de repositórios privados, é necessário que eles façam login [...]
% falar tbm da identificacao
% falar do your twydies

\section{"Curtir" Documentos}

to do
% \subsubsection

Limitações da Ferramenta

% nao importa código de branches
% nao liga para mudança nos arquivos
