\chapter[Fundamentação Teórica]{Fundamentação Teórica}
Este capítulo trata das referências utilizadas como base para o trabalho nas áreas de: mineração de repositórios de software, \textit{awareness} e gamificação.

\section{Mineração de repositórios de software} 
Mineração de repositórios de software é uma área que surgiu da necessidade de investigar fatos relevantes sobre projetos, produtos e pessoas no contexto de desenvolvimento de software; esta área ainda está amadurecendo devido ao aumento da quantidade de repositórios de código aberto. 

(HASSAN, 2008) Define que existem 5 tipos de repositórios:
\begin{enumerate}
\item Controle de versionamento de código, que são responsáveis por armazenar o histórico de desenvolvimento de um sistema
\item Repositórios de bug, que são responsáveis por manter informações de bugs e \textit{issues}
\item Arquivos de comunicação, como listas de e-mails, histórico de mensagens em chat interno, que podem rastrear discussões sobre projetos através do tempo
\item \textit{Logs} de integração 
\item Repositório de código, que é responsável por armazenar os códigos remotamente, como github\footnote{\url{http://github.com/}} ou sourceforge\footnote{\url{https://sourceforge.net/}}
\end{enumerate}
Recentemente houve um aumento em pesquisa utilizando dados de repositórios de software para obter importantes informações de vários aspectos da engenharia de software, tais como processos de software, produtividade do desenvolvedor, e evolução do software. O estudo de repositórios de software pode ser feito por diversas perspectivas, grande parte envolvendo predições. Outra conhecida abordagem na mineração de repositórios é a medida e análise da contribuição do desenvolvedor; isso pode ser muito útil no gerenciamento de equipes, estudos para aumento de salário, promoções, mudanças de equipe.
\section{Awareness} 
\textit{Awareness} é um termo que pode ser definido como um entendimento do trabalho e atividades de outros indivíduos, que de alguma forma criam um contexto para suas tarefas. Esse conceito é usado para certificar que atividades individuais tenham impacto como um todo e também para avaliar progresso individual como evolução do grupo inteiro. O termo envolve o conhecimento de quem está em volta, quais atividades estão sendo feitas, como a comunicação é feita, isso possibilita uma visão da rotina de trabalho diário no ambiente de trabalho. Awareness, facilita a ocorrência de interações informais e conexões espontâneas, que são aspectos muito importantes na manutenção de relacionamentos de trabalho
	Awareness é um ponto crucial para sistemas colaborativos e tem sido avaliado em diversos estudos nas últimas duas décadas. Os principais aspectos abordados nesses estudos eram relacionados a ferramentas com o objetivo de aumentar o conhecimento comum em equipes de desenvolvimento e o impacto que essas ferramentas tiveram nessas equipes. 
    Gutwin et al. investigou como desenvolvedores remotos mantinham o conhecimento interno do grupo, incluindo informações sobre quem está no projeto, em que parte do código estão trabalhando e quais são os seus planos. Esse tipo de informação é crucial para desenvolvedores remotos coordenarem seus esforços, adicionarem código sem prejudicar outros módulos, evitando retrabalho. Um estudo foi realizado com desenvolvedores opensource para saber quando os desenvolvedores precisam estar cientes dos outros, que informações eles buscam e como eles fazem para adquirir e manter esse conhecimento. Os autores entrevistaram quatorze desenvolvedores de de três diferentes projetos de opensource bem estabelecidos, examinaram emails e históricos de mensagens e analisaram artefatos do projeto como repositórios de código, páginas e documentação oficial e concluíram que eles mantinham não só conhecimento geral do time, mas também um contexto mais detalhado do que cada pessoa estava trabalhando e planejando trabalhar
\section{Gamificação e motivação}
Gamificação é definido como o contexto onde funcionários interagem e socializam por meio de um conhecimento comum e estratégia competitiva e divertida. Gamificação é mais do que um treinamento ultramoderno, é na verdade um modo de criar engajamento dos funcionários de uma forma que encoraje competitividade entre cada um e aplicar premiações e reconhecimento aos que se destacarem e isso pode abranger qualquer tópico, desde custos de logística, até o modo de desenvolvimento colaborativo de uma equipe de software, passando por atividades de gerência.
	Com gamificação é possível estabelecer metas baseadas em atividades comuns do cotidiano de uma empresa e estabelecer métricas de pontuação baseadas em conquistas e  utilizando as métricas definidas, é possível criar diversos tipos de premiações (recompensas virtuais). 
	É sabido que não podemos gerar motivação externamente a uma pessoa, entretanto a presença de possíveis prêmios, mesmo que sem valor real, estimula os funcionários a tentar alcançar os objetivos que irão conceder essas recompensas, desta forma estimulando também uma competitividade sadia entre cada membro da equipe ao mesmo tempo que o funcionário aumenta seu envolvimento e enganjamneto espontaneamente, ainda provocando um aumento na aprendizagem contínua da equipe. 